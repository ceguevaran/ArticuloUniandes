%% LyX 2.1.2 created this file.  For more info, see http://www.lyx.org/.
%% Do not edit unless you really know what you are doing.
\documentclass[spanish]{article}
\usepackage[T1]{fontenc}
\usepackage[utf8]{luainputenc}
\usepackage{geometry}
\geometry{verbose,tmargin=2.5cm,bmargin=2.5cm,lmargin=2.5cm,rmargin=2.5cm}
\usepackage{float}
\usepackage{amsmath}
\usepackage{graphicx}

\makeatletter

%%%%%%%%%%%%%%%%%%%%%%%%%%%%%% LyX specific LaTeX commands.
\floatstyle{ruled}
\newfloat{algorithm}{tbp}{loa}
\providecommand{\algorithmname}{Algoritmo}
\floatname{algorithm}{\protect\algorithmname}

%%%%%%%%%%%%%%%%%%%%%%%%%%%%%% User specified LaTeX commands.
\usepackage[ngerman,french,english]{babel}
\usepackage{subfigure}
\usepackage{amsmath}
\usepackage{apacite}
\date{}

\makeatother

\usepackage{babel}
\addto\shorthandsspanish{\spanishdeactivate{~<>}}

\begin{document}

\title{Evaluación de desempeño de un ecualizador espacial no lineal en ambientes
de comunicaciones móviles.\\
Performance evaluation of non linear space equalizer for mobile communications
environment.}

\maketitle
\author{\textbf{Carlos E. Guevara N.$^{1}$$^{*}$, César L. Niño$^{2}$$^{*}$\\ \\} 
\small{$^{1}$M.Sc. Ingeniería Electrónica, Profesor de Cátedra. carlos.guevara@javeriana.edu.co}\\ 
\small{$^{2}$PhD. Ingeniería Electrónica, Director de Maestría Ingeniería Electrónica Pontifica Universidad Javeriana. \\ cesar.nino@javeriana.edu.co}\\
\small{$^{*}$ Pontificia Universidad Javeriana, Bogotá D.C, Colombia}
}\\ \\
\small{\date{Recibido 14 de noviembre de 2014}}
\providecommand{\keywordsSP}[1]{\textbf{\textit{Palabras clave---}} #1}
\selectlanguage{spanish}
\begin{abstract} El problema de detección correcta de información transmitida a través de ambientes de comunicaciones móviles motiva a desarrollar un Ecualizador Espacial (SE) no lineal y evaluar su desempeño en estas condiciones. Para esto se construye la estructura general y se desarrolla un modelo matemático que permita el cálculo y actualización de los coeficientes del SE no lineal con el fin de garantizar un correcta detección de la información. Finalmente se evalúa el desempeño del ecualizador y se realizan comparaciones con un ecualizador lineal bajo diferentes condiciones de canal o ambientes de simulación y con distintas configuraciones de los ecualizadores.
\end{abstract}
\keywordsSP{Arreglo de antenas, Canal con desvanecimiento, Comunicaciones móviles, Ecualizador espacial, Filtro adaptativo, Filtro de Volterra,  LMS, MMSE, RLS.}
\providecommand{\keywordsEN}[1]{\textbf{\textit{Key words---}} #1}
\selectlanguage{english}
\begin{abstract} Reliable information detection on mobile communications environments lead us to develop a non linear Space Equalizer (SE) and evaluate its performance in different conditions. The general structure and the matematics model to estimate and update the equalizer coefficients are developed to achieve the goal of accurately detection of transmitted signals. Finally the SE performance is compared with the linear equalizer performance using different environment  conditions and equalizers settings.  \end{abstract}
\keywordsEN{Antenna arrays, Adaptive filter, Fading channel, LMS, MMSE, Mobile communications, RLS, Space equalizer, Volterra filter.}


\section{Introducción}

El creciente número de sistemas de comunicación móviles y los problemas
que estos generan en la transmisión y detección de la información,
resaltan la importancia de entender y comprender las características
de funcionamiento y operación de los mismos, las condiciones del canal
por el que se transmite, las propiedades de la señal recibida y los
problemas que se presentan en la detección de la información. Uno
de los mayores problemas de estos sistemas se refiere a la detección
confiable de información, debido a que la señal o señales recibidas
se han alterado o distorsionado al viajar a través del canal. 

Lo anterior destaca la importancia del tema de ecualización de señales
para poder suprimir o compensar problemas presentes en la señal recibida
tales como retardos, desfases, cambios en amplitud, interferencia
intersímbolo (ISI), interferencia co-canal (CCI), distorsión, ruido
o incluso superposición de múltiples versiones de la misma señal.
Específicamente el SE no lineal busca suprimir o compensar efectos
no deseados en la señal recibida sin incrementar la potencia de la
señal transmitida e igualmente la del ruido, debido a que el incremento
de potencia en el transmisor está condicionado por las características
físicas de los dispositivos que conforman el sistema. 

En \cite{Technology}, \cite{Rappaport},\cite{Paulraj1997} y \cite{Paulraj1997a}se
explica uno de los problemas que se pueden presentar en la propagación
de señales en sistemas de comunicaciones móviles, el desvanecimiento
o \emph{fading}. El \emph{fading} corresponde con variaciones de la
señal en amplitud, fase o retardo sobre un periodo de tiempo o la
distancia recorrida por la señal. Este fenómeno es causado por interferencia
entre dos o más versiones de la señal transmitida que llega al receptor
en tiempos diferentes e igualmente múltiples trayectorias por las
que viaja la señal producen efectos de desvanecimiento. En áreas urbanas
el\emph{ fading} es ocasionado porque la altura de las antenas móviles
es inferior a la de las estructuras que se encuentran alrededor, incluso
si hay línea de vista, ocurre el fenómeno de multi-camino debido a
reflexiones de la señal. Los tipos de \emph{fading }mencionados en
\cite{Rappaport} debido al tiempo de retardo de propagación multi-camino
son \emph{flat fading} y \emph{frequency selective fading}. 

La integración de las múltiples técnicas formuladas y desarrolladas
en \cite{Flury}, \cite{Rappaport}, \cite{Diniz2008} y \cite{Boccuzzi2008}
como las series de Volterra, ecualizadores lineales, filtrado adaptativo
y técnicas de optimización permitirán obtener como resultado final
el desarrollo y evaluación del SE propuesto para cumplir así el objetivo
este proyecto. El SE no lineal desarrollado será utilizado como se
dijo anteriormente, para compensar efectos no deseados en la señal
o señales recibidas ocasionados por las propiedades del canal y que
conllevan a una detección incorrecta y poco confiable.

El ecualizador no lineal propuesto utiliza un filtro polinomial cruzado
que es capaz de generar información adicional a partir de la que se
recibe originalmente en cada una de las antenas que componen el SE.
El contar con mayor información permitirá conseguir una mejor ecualización
de la señal y por ende una detección fiel de la información transmitida
originalmente.

El artículo está organizado como sigue, en la Sección \ref{sec:STE-no-lineal}
se describe el ambiente de operación, la estructura y el desarrollo
matemático para el SE no lineal propuesto. En la Sección \ref{sec:Resultados-de-Simulaciones}
se muestran, comparan y analizan los resultados de evaluación del
desempeño del ecualizador propuesto. Finalmente, en la Sección \ref{sec:Conclusiones}
se muestran las conclusiones.


\section{Ecualizador espacial no lineal\label{sec:STE-no-lineal} }

Se considera un sistema de comunicaciones inalámbrico que consta de
una antena transmisora y de $N$ antenas receptoras, es decir, un
sistema SIMO (\emph{Single Input Multiple Output}) donde cada antena
receptora recibe una versión diferente de la señal transmitida. Para
el sistema de comunicaciones se considerará un modelo de canal \emph{frequency
nonselective flat fading }con ruido aditivo\emph{. }

Para representar los efectos del desvanecimiento introducidos por
el canal, la señal transmitida se ve afectada por un factor multiplicativo
como se muestra en \cite{Proakis2012} y \cite{Yang2009} y que se
mantiene constante durante la duración o intervalo del símbolo; la
señal también se ve afectada por el ruido del canal. Para este proyecto
se evaluarán diferentes modelos de ruido cambiando su función de densidad
de probabilidad, se utilizará ruido con distribución de probabilidad
Normal o Gaussiana y también se evaluará el desempeño del ecualizador
si el modelo del ruido tiene una distribución Gaussiana Generalizada
\cite{Novey2010} variando el parámetro de forma. Tanto las funciones
del modelo del ruido como las funciones del modelo del \emph{fading
}son funciones complejas y se asumen estadísticamente independientes
de canal a canal e idénticamente distribuidas.

La descripción anterior del canal puede expresarse como 

\begin{equation}
y_{m}\left(n\right)=s\left(n\right)h_{m}+\eta_{m}\left(n\right);\ \ m=1,2,\dots,N
\end{equation}


donde $y_{m}\left(n\right)$ corresponde a la señal recibida en la
antena $m$, $s\left(n\right)$ corresponde a la señal o símbolo transmitido,
$h_{m}$ corresponde a\emph{ }la ganancia introducida por el canal
debido al desvanecimiento y $\eta_{m}\left(n\right)$ corresponde
al ruido aditivo que afecta a la señal o símbolo transmitido.

El SE no lineal propuesto está compuesto por un arreglo de $N$ antenas
receptoras donde cada una recibe una versión diferente de la señal
transmitida y cada una de éstas señales es una entrada a un \textbf{Filtro
Polinomial Cruzado} que procesará la información recibida y entregará
una señal ecualizada que es utilizada para la detección de la información.

El filtro propuesto basa su estructura en un filtro polinomial cuadrático
causal, discreto, invariante con el tiempo y de orden $M$ que está
descrito por medio de los primeros tres términos de la expansión en
series de Volterra, el primer término de la expansión corresponde
a un término de \emph{offset, }el segundo término hace referencia
a los coeficientes de un filtro FIR lineal e invariante con el tiempo
y el último término representa los coeficientes de un filtro cuadrático
homogéneo. En \cite{Budura},\cite{CONTAN2010}, \cite{Alper1963}
y \cite{Powers1985} proponen eliminar el término de \emph{offset}
en el modelo de un filtro polinomial a la hora de determinar el kernel
del filtro así como eliminarlo en la simulación. 

Se decidió utilizar un \emph{Symbol Spaced Equalizer} (SSE) para el
SE desarrollado, es decir, cada símbolo está representado únicamente
por una muestra. 

El filtro polinomial cruzado consiste de un filtro lineal asociado
a cada antena receptora del SE y un filtro cuadrático cruzado que
recibe dos señales diferentes de las antenas y produce una nueva señal
que será procesada. Las señales de salida tanto del filtro lineal
como del filtro cuadrático cruzado serán sumadas para obtener una
señal ecualizada. En general las señales recibidas en las antenas
del SE son señales complejas, se utilizará el complejo conjugado de
las señales cuando se multipliquen para ingresar al filtro polinomial
cruzado. La estructura general del filtro propuesto se muestra en
Fig. \ref{fig:EstructuraFiltroPolinomialCruzado}.

\begin{figure}[H]
\begin{centering}
\includegraphics[width=0.5\columnwidth]{\string"../../Desktop/ArticuloTesisMSc/img/Estructura Filtro Polinomial Cruzado\string".pdf}
\par\end{centering}

\protect\caption{\label{fig:EstructuraFiltroPolinomialCruzado}Estructura detallada
del SE propuesto, que consiste un arreglo de filtros lineales asociados
a cada una de las antenas y un filtro cuadrático cruzado que recibe
información de dos antenas diferentes y entrega una señal filtrada
de esta información. }
\end{figure}


A partir de la Fig. \ref{fig:EstructuraFiltroPolinomialCruzado} se
puede determinar la expresión matemática que describe completamente
el SE propuesto, para simplificar la notación y como se dijo anteriormente
sólo se trabajará con señales en el instante de tiempo $n$, $y_{m}$
se utilizará para denotar las señal $y_{m}\left(n\right)$

\begin{equation}
\begin{split}
y_{eq}&=y_{1}w_{1}+y_{2}w_{2}+\cdots+y_{N}w_{N}+\left(y_{1}y_{2}^{*}-h_{1}h_{2}^{*}\right)w_{1,2}\\
&+\cdots+\left(y_{1}y_{N}^{*}-h_{1}h_{N}^{*}\right)w_{1,N}+\cdots\\
&+\left(y_{N-1}y_{N}^{*}-h_{N-1}h_{N}^{*}\right)w_{N-1,N}
\end{split}
\end{equation}

lque puede reescribirse como

\begin{equation}
y_{eq}=\sum_{m=1}^{N}y_{k}w_{k}+\sum_{m=1}^{N-1}\sum_{n=m+1}^{N}\left(y_{m}y_{n}^{*}-h_{m}h_{n}^{*}\right)w_{m,n}
\end{equation}


De la expresión anterior se concluye que el número de señales que
se generan con información adicional en el SE está determinado por
la combinatoria{\tiny{} $\dbinom{N}{2}$}.

Utilizando notación vectorial para expresar la señal de salida $y_{eq}$
se tiene que

\begin{equation}
y_{eq}=\mathbf{y}^{T}\mathbf{w}
\end{equation}


donde 

\begin{equation}
\begin{split}
\mathbf{y} &= \left[y_{1}\ \ \ y_{2}\ \dots\ y_{N}\ \ \ y_{1}y_{2}^{*}-h_{1}h_{2}^{*}\ \cdots\right.\\ 
&\left.y_{1}y_{N}^{*}-h_{1}h_{N}^{*}\ \dots\ y_{N-1}y_{N}^{*}-h_{N-1}h_{N}^{*}\right]^{T}
\end{split}
\end{equation}

y

\begin{equation}
\mathbf{w=}\left[w_{1}\ \ \ w_{2}\ \dots\ w_{N}\ \ \ w_{1,2}\ \dots w_{1,N}\dots w_{N-1,N}\right]^{T}
\end{equation}


ambos vectores tienen dimensión {\tiny{}$\left[\left(N+\dbinom{N}{2}\right)\textnormal{x}1\right]$}. 

Con la estructura y representación matemática de un SE no lineal mostradas
anteriormente, se debe determinar los coeficientes óptimos de cada
uno de los filtros que componen el sistema. Se propone entonces realizar
una formulación MMSE (\emph{Minimum Mean Square Error}), que busca
estimar un vector de coeficientes $\mathbf{w}$ óptimo que minimice
la estimación del MSE, es decir encontrar un vector de coeficientes
que minimice la siguiente expresión

\begin{equation}
E\left\{ e^{2}\right\} =E\left\{ \left(s-\mathbf{y{}^{T}w}\right)^{2}\right\} 
\end{equation}


donde $s$ es la señal de respuesta deseada del sistema, esta información
se supone conocida por el emisor y receptor del sistema de comunicaciones,
por lo que es utilizada como señal de entrenamiento del ecualizador
antes de que este entre a funcionar en modo directo. 

El vector de coeficientes que garantiza la minimización de la función
de costo se deriva de la expresión anterior igualando la derivada
a cero y despejando el vector de coeficientes óptimos que está dado
por

\begin{equation}
\mathbf{w_{opt}}=\mathbf{C_{yy}^{-1}P}
\end{equation}


donde $\mathbf{C_{yy}}$ corresponde al valor esperado de la matriz
obtenida del producto entre el vector $\mathbf{y}$ definido anteriormente
con $\mathbf{y^{T}}$ y $\mathbf{P}$ que es un vector columna que
corresponde al valor esperado del vector resultante del producto entre
la señal de respuesta deseada $s$ y el vector $\mathbf{y}$, tanto
$\mathbf{C_{yy}}$ y $\mathbf{P}$ están definidos como sigue

\begin{equation}
\mathbf{C_{yy}=}E\left\{ \mathbf{yy{}^{T}}\right\} 
\end{equation}


y 

\begin{equation}
\mathbf{P=}E\left\{ s\mathbf{y}\right\} 
\end{equation}


$ $En las ecuaciones \ref{eq:10} y \ref{eq:P} se definen explicitamente
la matriz $\mathbf{C_{yy}}$ y el vector $\mathbf{P}$ respectivamente

{\small{}}
\begin{figure*}[t]
{\small{}
\begin{equation}
\mathbf{C_{yy}=}E\left\{ \left[\begin{array}{ccccc}
\left|y_{1}\right|^{2} & \cdots & y_{1}y_{N}^{*} & \cdots & y_{1}\left(y_{N-1}y_{N}^{*}-h_{N-1}h_{N}^{*}\right)^{*}\\
\vdots & \ddots & \vdots & \ddots & \vdots\\
y_{N}y_{1}^{*} & \cdots & \left|y_{N}\right|^{2} & \cdots & y_{N}\left(y_{N-1}y_{N}^{*}-h_{N-1}h_{N}^{*}\right)^{*}\\
\left(y_{1}y_{2}^{*}-h_{1}h_{2}^{*}\right)y_{1}^{*} & \cdots & \left(y_{1}y_{2}^{*}-h_{1}h_{2}^{*}\right)y_{N}^{*} & \cdots & \left(y_{1}y_{2}^{*}-h_{1}h_{2}^{*}\right)\left(y_{N-1}y_{N}^{*}-h_{N-1}h_{N}^{*}\right)^{*}\\
\vdots & \ddots & \vdots & \ddots & \vdots\\
\left(y_{N-1}y_{N}^{*}-h_{N-1}h_{N}^{*}\right)y_{1}^{*} & \cdots & \left(y_{N-1}y_{N}^{*}-h_{N-1}h_{N}^{*}\right)y_{N}^{*} & \cdots & \left|\left(y_{1}y_{2}^{*}-h_{1}h_{2}^{*}\right)\right|^{2}
\end{array}\right]\right\} \label{eq:10}
\end{equation}
}
\end{figure*}
{\small \par}

\begin{equation}
\label{eq:P}
\begin{split}
\mathbf{P}&=E\left\{ \left[sy_{1}\cdots sy_{N}\ s\left(y_{1}y_{2}^{*}-h_{1}h_{2}^{*}\right)\right.\right.\\
&\left.\left.\cdots s\left(y_{N-1}y_{N}^{*}-h_{N-1}h_{N}^{*}\right)\right]^{T}\right\} 
\end{split}
\end{equation}

donde la martriz $\mathbf{C_{yy}}$ es de dimensión {\tiny{}$\left[\left(N+\left(\begin{array}{c}
N\\
2
\end{array}\right)\right)\textnormal{x}\left(N+\left(\begin{array}{c}
N\\
2
\end{array}\right)\right)\right]$} mientras que los vectores $\mathbf{P}$ y $\mathbf{w_{opt}}$ son
de dimensión {\tiny{}$\left[\left(N+\left(\begin{array}{c}
N\\
2
\end{array}\right)\right)\textnormal{x}1\right]$}.


\subsection{Filtro Polinomial Cruzado Adaptativo}

Ahora se considera el caso de sistemas no lineales y variantes con
el tiempo\textbf{.} Para este tipo de sistema se propone desarrollar
un filtro que procese la señal de entrada para que la señal de salida
sea lo más cercana posible a la señal deseada, para esto el sistema
adapta los coeficientes del filtro continuamente usando la señal de
error como base de la actualización. Al igual que en el caso invariante
con el tiempo, se desarrollará el planteamiento para un filtro polinomial
cruzado adaptativo extendiendo los desarrollos para filtros truncados
de Volterra adaptativos como se muestran en \cite{Mathews2000}, \cite{Davila1987},
\cite{Mathews1988} y \cite{Diniz2008}. 

La salida del filtro $y_{eq}\left(n\right)$está dada por la siguiente
expresión

\begin{equation}
\begin{split}
y_{eq}\left(n\right)&=\sum_{m=1}^{N}y_{k}\left(n\right)w_{k}\left(n\right)+\\
&\sum_{m=1}^{N-1}\sum_{k=m+1}^{N}\left(y_{m}\left(n\right)y_{k}^{*}\left(n\right)-h_{m}\left(n\right)h_{k}^{*}\left(n\right)\right)w_{m,k}\left(n\right)
\end{split}
\end{equation}

y la señal de error está dada por 

\begin{equation}
e\left(n\right)=d\left(n\right)-y_{eq}\left(n\right)
\end{equation}


Se escogió el algoritmo \emph{Recursive Least Squares (RLS) }para
la construcción del filtro adaptativo ya que una de sus ventajas mencionada
en \cite{Diniz2008} y \cite{Mathews2000} es la rápida velocidad
de convergencia, incluso en escenarios donde los valores propios de
la matriz de autocorrelación del vector de entrada estén dispersos
en gran medida. El objetivo del filtro adaptativo RLS es actualizar
sus coeficientes de tal forma que su señal de salida $y_{eq}\left(n\right)$
durante el periodo de observación concuerde con la señal deseada tan
cerca como sea posible en el sentido de mínimos cuadrados.

La función de costo está definida por 

\begin{equation}
J\left(n\right)=\sum_{k=1}^{n}\lambda^{n-k}\left(d\left(k\right)-y_{eq}\left(n\right)\right)^{2}
\end{equation}


donde el parámetro $\lambda$ es un factor de peso exponencial que
controla la velocidad de convergencia del filtro adaptativo y está
en el rango $0<\lambda\le1$.

Para expresar la señal de salida del filtro utilizando notación vectorial
se tiene que

\begin{equation}
y_{eq}\left(n\right)=\mathbf{w\left(n\right)^{T}y\left(n\right)}
\end{equation}


donde 

\begin{equation}
\begin{split}
\mathbf{y}\left(n\right)&=\left[y_{1}\left(n\right)\ y_{2}\left(n\right)\ \dots\ y_{N}\left(n\right)\ y_{1}\left(n\right)y_{2}^{*}\left(n\right)-h_{1}h_{2}^{*}\ \dots\right.\\
&\left.y_{1}\left(n\right)y_{N}^{*}\left(n\right)-h_{1}h_{N}^{*}\dots y_{N-1}\left(n\right)y_{N}^{*}\left(n\right)-h_{N-1}h_{N}^{*}\right]^{T}
\end{split}
\end{equation}

y

\begin{equation}
\begin{split}
\mathbf{w}\left(n\right)&=\left[w_{1}\left(n\right)\ w_{2}\left(n\right)\ w_{3}\left(n\right)\ \dots\ w_{N}\left(n\right)\right.\\
&\left.w_{1,2}\left(n\right)\ \dots w_{1,N}\left(n\right)\dots w_{N-1,N}\left(n\right)\right]^{T}
\end{split}
\end{equation}

La solución al problema de minimización puede encontrarse derivando
$J\left(n\right)$ con respecto al vector de coeficientes $\mathbf{w\left(n\right)}$,
igualando las derivadas a cero y resolviendo las ecuaciones resultantes
para obtener el vector de coeficientes óptimo, este está dado por
la siguiente expresión 

\[
\mathbf{w\left(n\right)}=\mathbf{C^{-1}}\left(n\right)\mathbf{P}\left(n\right)
\]


donde 

\[
\mathbf{C^{-1}}\left(n\right)=\sum_{k=1}^{n}\lambda^{n-k}\mathbf{y}\left(k\right)\mathbf{y}^{T}\left(k\right)
\]


y 

\[
\mathbf{P}\left(n\right)=\sum_{k=1}^{n}\lambda^{n-k}\mathbf{y}\left(k\right)d\left(k\right)
\]


que corresponden a la matriz de autocorrelación del vector de entrada
ponderado exponencialmente y el vector de correlación cruzada entre
el vector de entrada y la señal de respuesta deseada ponderado exponencialmente
respectivamente. El algoritmo completo para el cálculo del filtro
adaptativo RLS es como se muestra en Algoritmo \ref{alg:AlgoritmoRLS}.

\begin{algorithm}[H]
\begin{itemize}
\item \textbf{Inicialización}\\
\textbf{
\[
\mathbf{C^{-1}}\left(0\right)=\frac{1}{\delta}\mathbf{I}
\]
\[
\mathbf{P}\left(0\right)=0
\]
\[
\mathbf{w}\left(0\right)=0
\]
}
\item \textbf{Iteración principal}\\
Para $n=1,2,3,\cdots$\\
\[
\epsilon\left(n\right)=d\left(n\right)-\mathbf{y}^{T}\left(n\right)\mathbf{\mathbf{w}}\left(n-1\right)
\]
\[
\mathbf{k}\left(n\right)=\frac{\mathbf{C^{-1}}\left(n-1\right)\mathbf{y}\left(n\right)}{\lambda+\mathbf{y}^{T}\left(n\right)\mathbf{C^{-1}}\left(n-1\right)\mathbf{y}\left(n\right)}
\]
\[
\mathbf{C^{-1}}\left(n\right)=\frac{1}{\lambda}\mathbf{C^{-1}}\left(n-1\right)-\frac{1}{\lambda}\mathbf{k}\left(n\right)\mathbf{y}^{T}\left(n\right)\mathbf{C^{-1}}\left(n-1\right)
\]
\[
\mathbf{\mathbf{w}}\left(n\right)=\mathbf{w}\left(n-1\right)+\mathbf{k}\left(n\right)\epsilon\left(n\right)
\]
\[
e\left(n\right)=d\left(n\right)-\mathbf{y}^{T}\left(n\right)\mathbf{w}\left(n\right)
\]

\end{itemize}
Donde $\delta$ es una constante pequeña positiva. 

\protect\caption{\label{alg:AlgoritmoRLS}Algoritmo RLS para el SE propuesto que actualiza
los coeficientes del filtro polinomial cruzado.}
\end{algorithm}



\section{Análisis de resultados\label{sec:Resultados-de-Simulaciones}}

Para determinar el desempeño del SE propuesto se realizarán diferentes
pruebas, estas involucraron cambiar la configuración del SE (variar
el número de antenas receptoras) y las condiciones del canal como
se mostrará a continuación. Asimismo para medir el desempeño del SE
propuesto se tomó como referencia el desempeño de un SE lineal y a
partir de esta comparación se evaluaron los resultados obtenidos. 

Para las pruebas realizadas se utilizaron dos, tres, cuatro y cinco
antenas para los dos SE (lineal y propuesto) y además se utilizó un
canal \emph{frequency nonselective flat fading }con ruido\emph{ }aditivo\emph{.
}Como se dijo anteriormente se decidió utilizar un \emph{Symbol Spaced
Equalizer, }donde cada símbolo está representado con una muestra y
se asume que los símbolos no tienen relación entre si. 

Las ganancias debido al desvanecimiento introducido por el canal se
asumen que se conocen perfectamente y tienen una distribución Gaussiana
con media cero y con varianza $\sigma_{h_{m}}^{2}$ idénticas y su
suma está normalizada a uno, esto es

\[
\sum_{m}E\left\{ \left|h_{m}\right|^{2}\right\} =1.
\]


Para el modelo del ruido aditivo en la primera prueba se utilizó una
distribución Gaussiana con media cero y con varianza $N_{0}/2$ para
generar la componente real y la componente imaginaria del ruido que
son independientes entre si. Como se dijo anteriormente tanto las
funciones del modelo del ruido como las funciones del modelo del \emph{fading
}son funciones complejas y se asumen estadísticamente independientes
de canal a canal e idénticamente distribuidas.

Con todas las consideraciones mostradas anteriormente, se realizaron
simulaciones de Montecarlo para evaluar el desempeño del SE. Como
criterios de desempeño se decidió utilizar basado en el estado del
arte, la curva de tasa de error de bit (BER) para diferentes valores
de relación señal a ruido por bit (SNR/bit) y el diagrama de constelación
para la señal original recibida en cada antena, para la señal ecualizada
usando un SE lineal y finalmente para la señal ecualizada usando el
SE propuesto.

El primer escenario de prueba consiste de un sistema compuesto por
un canal \emph{frequency nonselective flat fading }y ruido aditivo
con distribución de probabilidad Gaussiana y utilizando un esquema
de modulación BPSK. Los resultados obtenidos para este escenario se
muestran en la Fig. \ref{Figura51}, en esta se observa la tasa de
error de bit (BER) para diferentes valores de relación señal a ruido
por bit (SNR). 

\begin{figure*}
\centering \subfigure[]{\label{BERGG20}\includegraphics[width=0.48\textwidth]{\string"/Users/carlosegn/Desktop/Informe Final/Figuras/Spanish/BERGG20\string".jpg}}\ \ \subfigure[]{\label{BERGG15}\includegraphics[width=0.48\textwidth]{\string"/Users/carlosegn/Desktop/Informe Final/Figuras/Spanish/BERGG15\string".jpg}}\\

\centering\subfigure[]{\label{BERGG11}\includegraphics[width=0.5\textwidth]{\string"/Users/carlosegn/Desktop/Informe Final/Figuras/Spanish/BERGG11\string".jpg}}

\protect\caption{\label{CURVAS_BER}Comparación del desempeño de un SE lineal con el
SE propuesto utilizando dos, tres, cuatro y cinco antenas. Como medida
de desempeño se utiliza la tasa de error de bit (BER) para diferentes
valores de relación señal a ruido por bit (SNR/bit). Para este escenario
de prueba se utilizó un esquema de modulación BPSK, desvanecimiento
con distribución de probabilidad Gaussiana con media cero y varianza
normalizada a uno y se probaron diferentes modelos de ruido aditivo,
en \ref{BERGG20} se utilizó un modelo de ruido con distribución de
probabilidad Gaussiana con media cero y varianza $N_{0}/2$, en\ref{BERGG15}
se utilizó un modelo de ruido con distribución de probabilidad Gaussiana
generalizada con factor de forma $\rho=1.5$ y finalmente en \ref{BERGG11}
se utilizó un de ruido con distribución de probabilidad Gaussiana
generalizada con factor de forma 1.}
\end{figure*}


De los resultados obtenidos se puede observar que el SE no lineal
desarrollado es superior para todos los valores de SNR y de la misma
forma para todas las configuraciones (número de antenas) utilizadas
respecto al SE lineal. La razón fundamental para conseguir mejorar
el desempeño del SE lineal es el aumento de la dimensionalidad del
espacio del receptor que se genera al momento de realizar los productos
cruzados de las señales recibidas en cada antena, esta información
permite realizar una mejor ecualización de la señal y esto conlleva
a una correcta detección de la información transmitida originalmente.
Con el SE no lineal propuesto, se consigue una mejora en promedio
de 2 dB respecto al desempeño del SE lineal, una mejora considerable
se observa cuando se utiliza un número mayor de antenas receptoras.

El siguiente aspecto a evaluar en el ecualizador propuesto fue el
desempeño ante condiciones del canal mucho más agresivas que se representen
de mejor manera ambientes de comunicaciones móviles reales. Para esto
se decidió utilizar modelos de ruido aditivo que utilicen distribuciones
de probabilidad diferentes a la Gaussiana, se eligió una distribución
de probabilidad Gaussiana Generalizada y se varió el factor de forma
$\rho$ desde el caso Gaussiano cuando $\rho=2$ hasta el caso en
el que $\rho=1$ que se considera como el peor caso y que corresponde
a una distribución de Laplace. En la Fig. \ref{GG15} se muestra la
tasa de error de bit (BER) para diferentes valores de SNR, en general
el desempeño se degrada en comparación al caso Gaussiano. Sin embargo,
de la misma forma que ocurrió con el caso Gaussiano el desempeño del
SE mostrado en este trabajo es superior al del caso de un SE lineal,
se mantiene la ganancia en dB respecto al lineal que es de aproximadamente
2 dB. Se observa igualmente que el desempeño con cinco antenas sigue
siendo superior al resto de configuraciones. 

Por otro parte en la Fig. \ref{GG11} se observa la curva de desempeño
de BER vs SNR para un modelo de ruido con distribución Gaussiana Generalizada
con un factor de forma $\rho=1.1$ cercano al peor caso que pueda
considerarse. Evidentemente el desempeño en general es inferior a
los dos casos mostrados anteriormente, ya que para un valor de SNR
dado su valor de BER es mayor lo que genera una detección incorrecta
de la información. Sin embargo bajo estas condiciones la ganancia
en dB respecto al ecualizador lineal es en promedio superior a los
2 dB. Nuevamente el SE propuesto es superior en todas las diferentes
configuraciones. 

Por último se evaluó el desempeño del ecualizador para ambientes de
comunicaciones móviles que cambian en el tiempo utilizando un filtro
polinomial cruzado RLS con un factor de peso exponencial $\lambda=0.98$,
nuevamente se asumió que se conocía los coeficientes que describen
el modelo del canal y tienen las mismas características mencionadas
anteriormente, media cero y varianza $\sigma_{h_{m}}^{2}$, asimismo
el modelo del ruido tiene media cero y varianza unitaria, el esquema
de modulación utilizado fue BPSK, en la Fig. \ref{Adaptativo} se
muestran los resultados. Se puede observar la convergencia del ecualizador
propuesto para diferentes valores de SNR y para diferentes condiciones
de canal donde el modelo del ruido se hace cada vez más agresivo.
Las Figs. \ref{Adaptativo_a}, \ref{Adaptativo_b} y \ref{Adaptativo_c}
se muestran los resultados para un canal con modelo de ruido Gaussiano,
con modelo de ruido con distribución Gaussiano Generalizado con factor
de forma $\rho=1.5$ y con modelo de ruido con distribución Gaussiano
Generalizado con factor de forma $\rho=1.1$, en las tres gráficas
se observa que el desempeño se degrada a medida que las condiciones
del canal se vuelven más agresivas. Estos resultados respaldan los
obtenidos anteriormente donde también se mostraba que en general el
desempeño del ecualizador formulado se mantienen o degradan levemente
bajo diferentes condiciones de canal, lo que comprueba que el filtro
adaptativo desarrollado tiene un buen desempeño. 

\begin{figure*}
\centering \subfigure[]{\label{Adaptativo_a}\includegraphics[width=0.48\textwidth]{\string"/Users/carlosegn/Desktop/Informe Final/Figuras/Spanish/A20\string".jpg}}\ \ \subfigure[]{\label{Adaptativo_b}\includegraphics[width=0.48\textwidth]{\string"/Users/carlosegn/Desktop/Informe Final/Figuras/Spanish/A15\string".jpg}}\\

\centering\subfigure[]{\label{Adaptativo_c}\includegraphics[width=0.49\textwidth]{\string"/Users/carlosegn/Desktop/Informe Final/Figuras/Spanish/A11\string".jpg}}

\protect\caption{\label{Adaptativo}Curva de aprendizaje para el error cuadrático medio
del ecualizador propuesto para una configuración de dos antenas receptoras
y diferentes modelos de canal. En \ref{Adaptativo_a} se muestra la
curva de aprendizaje para un modelo de ruido con distribución Gaussiana.
En \ref{Adaptativo_b} se muestra la curva de aprendizaje para un
modelo de ruido con distribución de probabilidad Gaussiana generalizada
con factor de forma $\rho=1.5$. En \ref{Adaptativo_c} se muestra
la curva de aprendizaje para un modelo de ruido con distribución de
probabilidad Gaussiana generalizada con factor de forma $\rho=1.1$.}
\end{figure*}



\section{Conclusiones\label{sec:Conclusiones}}

En este trabajo se desarrolló un SE no lineal que utiliza un filtro
polinomial cruzado cuya mayor virtud es la de aumentar la dimensionalidad
del espacio del receptor a partir de las señales que se recibes originalmente
en el arreglo de antenas. La señales generadas tienen las mismas características
estadísticas de las señales recibidas y crece de manera exponencial
en función del número de antenas que componen el ecualizador. A medida
que se cuenta con mayor información el desempeño del ecualizador mejora
respecto al caso del ecualizador lineal. 

Con los resultados obtenidos se demuestra que el ecualizador desarrollado
tiene un desempeño superior respecto al ecualizador lineal en todos
los ambientes evaluados, sin importar la configuración o número de
antenas que conforman el ecualizador, diferentes valores de relación
señal a ruido y si las condiciones del canal se vuelven más agresivas,
haciéndolo más robusto. Las medidas de desempeño utilizadas permitieron
corroborar que ante diferentes condiciones de canal, con modelos de
ruido mucho más agresivos el ecualizador desarrollado mantenía su
desempeño casi sin alterarse y de la misma manera continuaba siendo
superior al del ecualizador lineal. Respecto al número de antenas
receptoras del SE propuesto, se concluye que entre mayor sea el número
el ecualizador presenta un mejor desempeño y esto es debido a que
se genera información adicional que incrementa en función del número
de antenas, permitiendo de esta manera una mejor ecualización de la
señal y garantizando de esta manera una correcta detección de la información
transmitida originalmente. Estos resultados soportan los desarrollos
teóricos realizados para este proyecto. 

Finalmente, se demostró convergencia del ecualizador propuesto ante
sistemas de comunicaciones que varían lentamente con el tiempo y este
debe adaptarse a las condiciones del canal para que la señal de error
sea la mínima y su desempeño sea adecuado, de igual manera cómo sucedió
en el caso cerrado, su desempeño se mantiene prácticamente igual a
medida que las condiciones del canal se vuelven más agresivas.

Los desarrollos anteriores pueden extenderse para formular un \emph{Fractionally
Spaced Equalizer }donde cada símbolo está representado por más de
una muestra y estas muestras tienen relación entre sí, ya que pertenecen
al mismo símbolo. Como se mostró en este trabajo a medida que la información
disponible en el ecualizador su desempeño mejora, razón por la cual
al tener disponible mayor número de información se puede conseguir
que su desempeño sea aun mejor. 

\bibliographystyle{apacite}
\bibliography{\string"/Users/carlosegn/Desktop/Informe Final/References\string"}

\end{document}
